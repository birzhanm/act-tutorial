\documentclass{report}
\title{Tutorial on Applied Category Theory}
\begin{document}
\maketitle
\chapter{Lesson 1}
\textbf{Questions}\\
\begin{enumerate}
	\item Why Applied Category Theory?
	\item What is a category?
	\item How do categories arise?
\end{enumerate}
\section*{Why Applied Category Theory?}
A study of applied category theory provides seemingly contradictory experiences. It lets own to work with quite abstract objects, as well as it has a promising potential to shed some light on workings of various systems around us. The latter is also known as compositionality, or when an object is much more than sum of its parts.

\section*{What is a category?}
Category can be thought as a class. For example it is common for people to say
\begin{enumerate}
	\item Let us consider $Top$, the category of topological spaces.
	\item Look at $Vect$, a category of vector spaces over some field.
\end{enumerate}

\section*{How do categories arise?}
Historically, category theory arose as an observation that seemingly unrelated mathematical objects could exhibit a common theme.
\end{document}
